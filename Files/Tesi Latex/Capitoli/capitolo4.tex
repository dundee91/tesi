\chapter{Lavori simili}

Ogni regione è stata chiamata per legge a sviluppare il proprio sistema di fatturazione e, eventualmente, a porsi come nodo locale di interscambio per le fatture degli enti locali.
In Regione Lazio il sistema \textbf{HUB} \cite{hub} svolge appunto questo ruolo, esso prevede un sito per l'adesione al sistema da parte delle \Gls{ppaa} e dei privati a tre tipi di servizio: fattura attiva, fattura passiva, conservazione.
La Regione Emilia-Romagna predispone anch'essa un sistema di interscambio regionale chiamato \textbf{NoTI-ER} \cite{notier}, il quale una volta ricevuta la fattura dallo \Gls{sdi} prima la converte nel formato europeo PEPPOL \cite{peppol}, poi la invia al software \textbf{SICIPA-ER} (l'equivalente del nostro Fatto) e al sistema di conservazione Par-ER.
Troviamo un sistema di interscambio anche per la Regione Toscana, chiamato \textbf{fERT}, che oltre alla comunicazione tra \Gls{sdi} e Ente, si occupa anche della comunicazione con la \Gls{pcc}.
In Regione Marche il ruolo di intermediario svolto da \Gls{imm} \cite{intermedia} è quello di mettere in comunicazione lo \Gls{sdi}, con Fatto e Fatto con il sistema di protocollo, quindi presiede a tutto il processo di fatturazione fino al momento dell'accettazione/rifiuto.
Successivamente le comunicazioni con la \Gls{pcc} passano per il software della contabilità Siagi, il quale prima di inviare i dati relativi ai pagamenti, preleva le informazioni necessarie (associazioni tra fatture e decreti, tempi di accettazione e contabilizzazione, etc) da Fatto.

%\chapter{Deploy e manutenzione}

%\section{Le fasi di produzione}

%\section{Compilazione}

%\section{Pubblicazione}

%\section{Cattura e gestione degli errori}
