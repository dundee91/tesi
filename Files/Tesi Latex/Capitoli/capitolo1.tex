% 2

\chapter{Bilancio di un'azienda}

L’analisi di bilancio, è uno strumento tecnico, che permette di studiare i risultati d’esercizio derivanti dalle attività aziendali, focalizzando l’attenzione su:
\begin{itemize}
 \item situazione economica: condizioni che determinano la produttività dell’azienda e costituisce la capacità di determinare utili d’esercizio;
 \item situazione finanziaria: condizioni che determinano la capacità di far fronte ai debiti a breve scadenza e all’accesso a prestiti di finanziamento;
 \item situazione patrimoniale: condizioni che determinano il rapporto tra le immobilizzazioni e i debiti verso terzi.
\end{itemize}
In altre parole, con lo strumento dell’analisi di bilancio, si cerca di individuare nelle diverse aree (economica, finanziaria e patrimoniale) i primi spunti di riflessione per l’analisi della gestione.

Il bilancio di un'azienda, nella sua forma più semplice, è formato da due documenti: lo \textbf{Stato Patrimoniale}, che rappresenta in modo sintetico la composizione qualitativa e quantitativa del patrimonio della società al giorno della chiusura dell’esercizio, e il \textbf{Conto Economico}, che espone il risultato economico dell’esercizio attraverso la rappresentazione dei costi e degli oneri sostenuti, nonché dei ricavi e degli altri proventi conseguiti nell’esercizio;.


% 2.1

\section{Stato Patrimoniale}

Lo Stato Patrimoniale, definito dall'art. 2424 del Codice Civile, rappresenta la situazione patrimoniale ad una certa data di un'impresa ed è suddiviso in due sezioni contrapposte chiamate Attività e Passività.

In particolare, le attività sono distinte secondo criteri di realizzabilità ed esigibilità, in :
\begin{itemize}
 \item crediti verso soci per versamenti ancora dovuti, con separata indicazione della parte già richiamata;
 \item immobilizzazioni: rappresentano gli investimenti destinati per un periodo superiore a 12 mesi, a trasformarsi in denaro. Sono iscritte al netto dei fondi di ammortamento;
 \item attivo circolante: rappresentano gli investimenti destinati per un periodo inferiore a 12 mesi, a trasformarsi in denaro. Iscritto al netto di eventuali fondi di svalutazione;
 \item ratei e riscontri;
\end{itemize}

Le passività, nell’analisi di bilancio, vengono distinte secondo le scadenze di pagamento in:
\begin{itemize}
 \item patrimonio netto;
 \item fondi per rischi e oneri;
 \item trattamento di fine rapporto di lavoro subordinato;
 \item debiti;
 \item ratei e riscontri;
\end{itemize}

Si possono individuare due criteri di riclassificazione dello stato patrimoniale per acquisire migliori informazioni sulle dinamiche aziendali: il criterio funzionale e quello finanziario.

% 2.1.1

\subsection{Riclassificazione Funzionale (Operativo)}

Secondo il criterio funzionale invece le attività (impieghi) e le passività (fonti) sono riclassificate in base all’area gestionale di appartenenza: area caratteristica/operativa (nella quale ricomprendere se marginale anche quella accessoria), comprendente tutti i valori attinenti il core business; area finanziaria, comprendente tutti i valori relativi alla negoziazione di liquidità.\\
Gli impieghi sono pertanto suddivisi in:
\begin{itemize}
\item attività operative: assets materiali e immateriali, crediti operativi, rimanenze, ratei e risconti;
\item attività finanziarie: investimenti finanziari (a breve e a medio-lungo), crediti finanziari e disponibilità liquide.
\end{itemize}

Le fonti sono invece suddivise in:
\begin{itemize}
\item patrimonio netto: grandezza non riconducibile né all’area operativa né a quella finanziaria;
\item passività operative: fondi rischi ed oneri, debiti operativi e ratei e risconti;
\item passività finanziarie: ovvero i debiti finanziari a prescindere dalla scadenza.
\end{itemize}

Lo stato patrimoniale classificato secondo la logica funzionale mira a verificare l’equilibrio fra investimenti e fonti di finanziamento, e quindi di ausilio a sviluppare l’analisi della solidità

% 2.1.2

\subsection{Riclassificazione Finanziario}

Con il criterio finanziario le attività (impieghi) sono classificate e raggruppate secondo il loro grado di liquidabilità, ovvero in funzione della loro capacità di trasformarsi in liquidità in tempi più o meno rapidi, mentre le passività (fonti) in base alla loro durata temporale, ovvero in base alla loro velocità di estinzione.

L’arco temporale preso a riferimento con termine congruo per circoscrivere il breve dal medio-lungo termine corrisponde a 12 mesi.

Gli impieghi sono pertanto suddivisi, in funzione alla loro effettiva possibilità di trasformarsi in liquidità, in:
\begin{itemize}
 \item attività correnti, atte ad essere liquidate in un arco temporale inferiore a 12 mesi, ovvero assets destinati alla vendita entro 12 mesi, attività finanziarie detenute a scopo di negoziazione, crediti in scadenza entro 12 mesi, rimanenze (per la parte che presenta un tasso di rotazione inferiore a 12 mesi), liquidità, ratei e risconti;
 \item attività non correnti, destinate a rimanere vincolate nel medio-lungo periodo, ovvero assets materiali, immateriali e finanziarie (eccetto quelle destinate alla vendita nel breve termine), crediti con scadenza oltre il 12 mesi, rimanenze (per la parte che presenta un tasso di rotazione inferiore a 12 mesi).
\end{itemize}

Le fonti sono invece suddivise in:
\begin{itemize}
 \item patrimonio netto, grandezza vincolata e quindi fonte di lungo periodo;
 \item passività correnti, destinate al rimborso entro i 12 mesi, ossia: debiti a breve (comprese le rate a breve di finanziamenti a medio-lungo termine), ratei e risconti passivi, fondi rischi ed oneri (per la parte che avrà manifestazione finanziaria nel breve periodo);
 \item passività non correnti, con scadenza superiore a 12 mesi, ossia: debiti a medio-lungo, risconti passivi pluriennali, fondi rischi ed oneri (per la parte che avrà manifestazione finanziaria oltre 12 mesi).
\end{itemize}

Lo stato patrimoniale classificato secondo la logica finanziaria permette di verificare la capacità dell’azienda di far fronte ai propri impegni di breve periodo con impieghi di egual durata (capitale circolante), ed è pertanto propedeutico all’analisi della liquidità;
\\
\\
\\

% 2.2

\section{Conto Economico}

Il Conto Economico, definito dall'art. 2425 del Codice Civile, è l'elenco, ordinato per categorie, dei costi e dei ricavi di competenza dell'esercizio, ossia di competenza di quel lasso di tempo intercorrente tra la data di riferimento del bilancio attuale e quella del bilancio precedente.

Si intendono ricavi le vendite dei propri beni, gli interessi attivi o i fitti attivi. 
Sono, invece, esempi di costi gli acquisti, le utenze, le spese del personale, i fitti passivi le imposte e le tasse. 

In tale contesto è bene far presente che l'Iva, ancorché un'imposta, non rappresenta un costo per la società. \\
Si tratta, infatti, di un debito o un credito che l'azienda ha nei confronti dell'Erario nel momento in cui compie una vendita o un acquisto e, per questo motivo, trova allocazione nel passivo o nell'attivo dello Stato Patrimoniale.

Esso ha una struttura a forma scalare e una classificazione dei costi per natura (invece che per destinazione). È formato da quattro sezioni (individuate con le prime lettere dell'alfabeto), più alcune voci che illustrano il risultato d'esercizio, ante e dopo le imposte.

Sezioni che compongono il conto economico:
\begin{itemize}
 \item valore della produzione;
 \item costi della produzione;
 \item proventi ed oneri finanziari;
 \item rettifiche di valore di attività e passività finanziarie.
\end{itemize}

Il Conto Economico può assumere diverse forme e configurazioni. Le riclassificazioni più utilizzate, per l’analisi di bilancio sono:
\begin{itemize}
\item il conto economico a margine di contribuzione, che si basa sulla suddivisione dei costi operativi tra costi fissi e costi variabili;
\item il conto economico a costo del venduto, che si basa sulla suddivisione dei costi operativi tra costi diretti e costi indiretti;
\item il conto economico a valore aggiunto, che si basa sulla suddivisione dei costi operativi tra costi relativi alle risorse esterne e costi relativi alle risorse interne.
 \end{itemize}

Nel software è stata presa in considerazione solo la riclassificazione a valore aggiunto.

% 2.2.2

\subsection{Riclassificazione al Valore Aggiunto}

Il valore aggiunto, quale differenza tra ricavi operativi e costi operativi sostenuti per l’acquisto di risorse esterne, esprime la capacità dell’azienda di creare ricchezza per remunerare i fattori produttivi e i diversi portatori di interesse.

In particolare tale margine deve essere in grado di remunerare:
\begin{itemize}
\item il personale - costo del personale;
\item gli investimenti - ammortamenti e svalutazioni;
\item i finanziatori esterni - componenti finanziarie;
\item gli eventi straordinati - componenti straordinarie;
\item l’Amministrazione finanziaria - imposte.
\end{itemize}

Deve infine garantire un’adeguata remunerazione, tramite la distribuzione del risultato d’esercizio, ai soci e permettere con l’utile residuo non distribuito un adeguato autofinanziamento.


% 2.3

\section{Forecast}

Il Forecast è un prospetto simile a un normale bilancio. A differenza di quest’ultimo, che registra valori a consuntivo, il Forecast è una tabella che contiene valori di natura previsionale. Nello specifico, dà una previsione che guarda avanti di qualche mese.

L’impostazione e lo scopo del Forecast sono simili a quelli del budget. Cambia però il periodo di analisi. Infatti, mentre il budget è una previsione fatta sull’anno successivo, il Forecast è una previsione sull’anno in corso.

