%\chapter{Introduzione}
\chapter*{Introduzione}
\addcontentsline{toc}{chapter}{Introduzione}
\markboth{INTRODUCTION}{}

L'analisi di bilancio è una delle procedure attraverso cui vegono determinati i fattori che determinano i guadagni, le risorse attraverso cui l'azienda ripaga i debiti e il grado di indebitamento.
In questa tesi verrà mostrato il software da me realizzato per il processo di analisi di bilancio attraverso la riclassificazione dello Stato Patrimoniale, del Conto Economico e del forecast per indici di previsioni.\\
Inizialmente verrano descritte le tecnologie utilizzate, successivamente il contesto economico dell'analisi di bilancio, poi sarà discussa la fase di analisi e progettazione che ha preceduto la realizzazione del software, descrivendo attraverso diagramma \Gls{uml} \cite{uml} i vari requisiti, vincoli e funzionalità del prodotto. \\
Successivamente si entrerà nel merito della realizzazione di tutte le funzionalità richieste. \\
In conclusione si farà riferimento ad eventuali sviluppi futuri che potrebbero essere implementati nel software realizzato.