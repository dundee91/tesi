\chapter*{Conclusione}
\addcontentsline{toc}{chapter}{Conclusione}
\markboth{CONCLUSION}{}

La mia collaborazione alla realizzazione di questo progetto mi ha permesso di valutare in un contesto concreto l'efficacia sia delle metodologie di analisi e progettazione, sia delle tecnologie di sviluppo utilizzate.
Essendo stato partecipe inoltre ad alcuni processi decisionali nel ruolo di referente tecnico, ho potuto confermare l'importanza di una buona scrittura dei requisiti funzionali e non, accompagnata da una gestione delle attività e delle scadenze.

Dal punto di vista del personale amministrativo non ci sono state difficoltà nell'apprendere l'utilizzo del software, anche grazie agli accorgimenti presi per il front-end e alla continua disponibilità, mia e dei miei referenti.
In ogni caso alcune problematiche di dominio sono venute fuori solo successivamente al rilascio della prima versione del software, ciò ha comportato un continuo lavoro di rianalisi e riadattamento che ha interessato tutte le parti responsabili dei software coinvolti nel processo di fatturazione.

Parlando di vantaggi al cittadino, da un lato si sono ottenuti dei tempi certi per quanto riguarda l'accettazione e il rifiuto (garantiti dalla spietatezza dei contatori informatici) e si è indirettamente dato uno standard ai vari software proprietari e non per la gestione delle fatture.
Da un altro lato però sono aumentati gli oneri per i fornitori, perché anche ammettendo che si riesca a realizzare autonomamente l'xml della fattura da inviare o che il vecchio software di fatturazione in proprio possesso sia stato aggiornato senza ulteriori spese, si deve far fronte alle spese per la conservazione digitale, ovvero alle spese per un software di conservazione aderente agli standard e alle marche temporali digitali.

Tutti questi problemi spero si risolvano nel futuro, scegliendo magari di gestire la conservazione direttamente lato \Gls{sdi}, con un maggiore sforzo implementativo da parte del Ministero.

Complessivamente mi ritengo soddisfatto di quanto fatto finora, ma sono consapevole che la realizzazione del progetto non è che un passo verso una complessiva digitalizzazione e sostanziale modifica del funzionamento della \Gls{pa}.

I futuri sviluppi che interesseranno il sistema di fatturazione saranno legati al prossimo rilascio da parte della \Gls{pcc} dei web service per lo scambio delle informazioni relative ai pagamenti effettuati.
Non è esclusa tuttavia anche una possibile adozione degli standard PEPPOL, relativi al processo di fatturazione o anche una parziale ridefinizione di alcuni processi amministrativi digitali.